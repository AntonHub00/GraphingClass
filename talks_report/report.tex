\documentclass{article}
%"Plantilla" del documento

\usepackage[spanish]{babel}
%Reconoce caracteres clave como pueden ser acentos
%Formatos en español como pueden ser fechas ser fechas
%Reconoce reglas tipográficas
\usepackage[T1]{fontenc}
\usepackage[utf8]{inputenc}
%Permite usar "á" directamente en lugar de \'a, entre otras cosas
%No recomendado si se comparte el archivo fuente o si se trabaja en equipo

\title{Reporte de conferencias Tékhne 2018}
\author{Antonio Emiko Ochoa Adame}

\begin{document}

\maketitle

\section{Conferencia 1: ``Digital Transformation and Software/Requirements  Engineering''}
%Mi primer documento en \LaTeX.
Esta conferencia trató el tema de la transformación digital, lo que esta implica y las ventajas que trae consigo. Una empresa debe que tener una transformación digital si quiere seguir compitiendo con las demás empresas que están allá afuera, las cuales son potencias en el sector.

Sea de software o no, una empresa necesita llevar a cabo esta transformación ya que debe adaptarse al mercado tanto actual como futuro; en este caso, la tecnología a través de las web apps, aplicaciones móviles y otros medios similares está dominando el mercado, por lo tanto, es necesaria esta adaptación si lo que se desea es mantenerse en el juego por mucho más tiempo.

Como ejemplo se puede mencionar el caso de Netflix vs Blockbuster. Netflix empezó a buscar nuevas formas de introducirse en el mercado; una de ellas fue la renta de películas por correo. Esto no está directamente relacionado con la tranformación digital, o por lo menos eso parece. Netflix, en su búsqueda por estas nuevas estrategias llegó antes que Blockbuster a la transformación digital donde pudo anticiparse al ``futuro'' para así poder superar a su competidor, el cual actualmente se considera prácticamente extinto.

Otro ejemplo de lo anterior es Uber. Esta empresa logró introducirse en un mercado difícil, dominado prácticamente por los taxis y sus sindicatos. El gran éxito de dicha empresa de debe (entre otras cosas como el servicio y su trato hacia el cliente) al hecho de que el núcleo de la empresa está basado en software, ya que a través de este se realizan las transacciones y contacto, por ejemplo.

\section{Conferencia 2: ``Green Route, la aplicación para una movilidad inteligente''}
\end{document}
